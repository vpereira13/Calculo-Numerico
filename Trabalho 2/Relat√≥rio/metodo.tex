\chapter{Simpson 1/3 composto}
A regra de Simpson baseia-se em aproximar a integral definida pela área sob arcos de parábola que interpolam a função.

Podemos ver que a fórmula de Simpson fornece uma boa aproximação se o intervalo de integração \([ a , b ]\) for pequeno, o que não acontece na maior parte do tempo. A solução óbvia é dividir o intervalo de integração em intervalos menores, aplicar a fórmula de Simpson para cada um destes e somar os resultados.

Ou seja, a regra de Simpson 1/3 composto nada mais é que aplicar a regra de Simpson 1/3 a cada dois intervalos, mas é importante notar que é preciso ter uma quantidade par de subintervalos, ou seja, uma quantidade ímpar de pontos.
