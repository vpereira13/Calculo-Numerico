\chapter*[Introdução]{Introdução}
\addcontentsline{toc}{chapter}{Introdução}
Em cálculo, algumas integrais definidas não são tão simples de calcular ou até
mesmo não se tem a função propriamente dita, só alguns valores para a função em
alguns pontos. As integrações numéricas formas para aproximar uma intragral que tenha as
caracteristicas anteriores.

No caso desse segundo trabalho de Cálculo Numérico tivemos que implementar um
programa em Matlab/Octave ou em C que implementasse o método de Simpson 1/3
composto para calcular uma integral dada.

A regra de Simpson 1/3 composto nada mais é que aplicar a regra de Simpson 1/3 a
cada dois intervalos, importante notar que é preciso ter uma quantidade par de
subintervalos, ou seja, uma quantidade ímpar de pontos.
