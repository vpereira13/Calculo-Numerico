\chapter*[Introdução]{Introdução}
\addcontentsline{toc}{chapter}{Introdução}
Em cálculo, algumas integrais definidas não são tão simples de calcular, para
isso foram desenvolvidos métodos numéricos para calcular essas integrais
difíceis. As integrações numéricas são formas para aproximar uma intragral que
tenha a caracteristica ou até mesmo quando não se tem a função propriamente
dita, mas tem alguns valores da função em alguns pontos.

No caso desse segundo trabalho de Cálculo Numérico, tivemos que implementar um
programa em Matlab/Octave ou em C que calculasse a integral de uma dada função
utilizando o método de Simpson 1/3 composto.

A regra de Simpson 1/3 composto nada mais é que aplicar a regra de Simpson 1/3 a
cada dois intervalos, é importante notar que é preciso ter uma quantidade par de
subintervalos, ou seja, uma quantidade ímpar de pontos.
