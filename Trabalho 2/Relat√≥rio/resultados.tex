\chapter{Resultados}
\section{Primeira etapa --- Simpson 1/3 composto}
\label{sec:resultadoSimpson}
Inicialmente nos foi pedido para utilizar o programa que criamos para calcular
\(F(1)\times F(2)\), em que \(F(1) = I(f)(1) - 0.45\) e  \(F(2) = I(f)(2) -
0.45\) e verificar que o resultado dessa multiplicação é menor que zero, ou
seja, que existe uma raiz nesse intervalo (\([1, 2]\)).

Para realizar esse teste, primeiro temos que calcular a \(I(f)(1)\) e
\(I(f)(2)\), para isso utilizaremos o programa \texttt{main\_simpson} e dois
arquivos auxiliares de entrada, o arquivo
\texttt{simpson1.in}(\ref{app:input_simpson1}) e o
\texttt{simpson2.in}(\ref{app:input_simpson2}), ambos presente na pasta
\texttt{inputs/}. Rodando o programa \texttt{main\_simpson} da forma que foi
explicada na subseção~\ref{subsec:ex_com}, obtivemos as seguintes saídas,
respectivamente:

\begin{minted}[autogobble,breaklines,linenos,frame=lines,fontsize=\footnotesize]{bash}
	make run_simpson < inputs/simpson1.in
	0.3413447460685430
\end{minted}

\begin{minted}[autogobble,breaklines,linenos,frame=lines,fontsize=\footnotesize]{bash}
	make run_simpson < inputs/simpson2.in
	0.4772498680518208
\end{minted}

Agora fazendo a operação solicitada:
\begin{align*}
F(1) &=  0.3413447460685430 - 0.45\\
     &= -0.10865525393145703\\
F(2) &=  0.4772498680518208 - 0.45\\
     &=  0.02724986805182078\\
F(1) \times F(2) &= -0.10865525393145703 * 0.02724986805182078\\
                 &= -0.0029608413327692853\\
\end{align*}
\hfill\(\blacksquare\)

\section{Segunda etapa --- Método de Newton}
Nessa segunda etapa foi solicitado que aproximemos a raiz que foi provada a
existência na seção anterior, a seção \ref{sec:resultadoSimpson}, utilizando o
programa que fizemos.

Com chute inicial de \(0.5\) e precisão de \(10^{-10}\)Também utilizando a
arquivo auxiliar de entrada, no caso o \texttt{newton1.in}, obtivemos a seguinte
saída:

\begin{minted}[autogobble,breaklines,linenos,frame=lines,fontsize=\footnotesize]{bash}
	make run_newton < inputs/newton1.in
	Numero de iteracoes do metodo de Newton: 7
	1.6448536269391639
\end{minted}

Ou seja, a aproximação que obtivemos utilizando o método de Newton para a raiz
foi o ponto \(1.6448536269391639\) e esse ponto com a precisão solicitada foi
encontrado após sete iterações.
