\chapter*{Conclusão}
\addcontentsline{toc}{chapter}{Conclusão}
Como foi pedido, utilizamos a linguagem C como base para fazer o trabalho. Implementamos também algumas bibliotecas para nos auxiliar e modularizar o código, contendo métodos específicos de cada regra, Simpson 1/3 e de Newton, e uma de base para as funções que precisávamos calcular.

Na prática, após estudar a matéria, não tivemos tantas dificuldades sobre como resolver o problema proposto, o grupo conseguiu implementar os códigos de Simpson 1/3 e de Newton e executá-los com as entradas necessárias para obter os resultados com uma boa precisão.

Para a realização dos testes pedidos, utilizamos arquivos de entrada que podem ser vistos no apêndice \ref{ch:inputs}, nos quais os utilizados para a regra de Simpson 1/3 composta fizemos com 10000 subdivisões, para obter uma boa precisão, já o utilizado para o método de Newton, está como foi pedido, precisão de \(10^{-10}\) e com chute inicial \(x_0 = 0.5\).

Com relação aos resultados obtidos no capítulo~\ref{sec:resultadoSimpson} podemos verificar a eficiência da regra de Simpson 1/3 composta e, também, do nosso trabalho.