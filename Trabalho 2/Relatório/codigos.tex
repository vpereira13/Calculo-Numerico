% ---
% Inicia os apêndices
% ---
\begin{apendicesenv}

% Imprime uma página indicando o início dos apêndices
\partapendices

\chapter{Códigos Fonte}
\label{ch:codigos_fonte}
	\section{Programas principais}
		\subsection{\texttt{main\_simpson.c}}
		\label{subsec:main_simpson.c}
		\inputminted[autogobble,breaklines,linenos,frame=lines,tabsize=4,obeytabs=true,fontsize=\footnotesize
		]{c}{source_codes/main_simpson.c}

		\subsection{\texttt{main\_newton.c}}
		\label{subsec:main_newton.c}
		\inputminted[autogobble,breaklines,linenos,frame=lines,tabsize=4,obeytabs=true,fontsize=\footnotesize
		]{c}{source_codes/main_newton.c}

	\section{Programas auxiliares}
		\subsection{\texttt{Makefile}}
		\label{subsec:Makefile}
			\inputminted[autogobble,breaklines,linenos,frame=lines,tabsize=4,obeytabs=true,fontsize=\footnotesize
			]{c}{source_codes/Makefile}

		\subsection{\emph{Header} da biblioteca de Funções (\texttt{funcoes.h})}
		\label{subsec:funcoes.h}
			\inputminted[autogobble,breaklines,linenos,frame=lines,tabsize=4,obeytabs=true,fontsize=\footnotesize
			]{c}{source_codes/funcoes.h}

		\subsection{Implementação da biblioteca de Funções (\texttt{funcoes.c})}
			\label{subsec:funcoes.c}
			\inputminted[autogobble,breaklines,linenos,frame=lines,tabsize=4,obeytabs=true,fontsize=\footnotesize
			]{c}{source_codes/funcoes.c}

		\subsection{\emph{Header} da biblioteca de Simpson (\texttt{simpson.h})}
		\label{subsec:simpson.h}
			\inputminted[autogobble,breaklines,linenos,frame=lines,tabsize=4,obeytabs=true,fontsize=\footnotesize
			]{c}{source_codes/simpson.h}

		\subsection{Implementação da biblioteca de Simpson (\texttt{simpson.c})}
			\label{subsec:simpson.c}
			\inputminted[autogobble,breaklines,linenos,frame=lines,tabsize=4,obeytabs=true,fontsize=\footnotesize
			]{c}{source_codes/simpson.c}

		\subsection{\emph{Header} da biblioteca de Newton (\texttt{newton.h})}
		\label{subsec:newton.h}
			\inputminted[autogobble,breaklines,linenos,frame=lines,tabsize=4,obeytabs=true,fontsize=\footnotesize
			]{c}{source_codes/newton.h}

		\subsection{Implementação da biblioteca de Newton (\texttt{newton.c})}
			\label{subsec:newton.c}
			\inputminted[autogobble,breaklines,linenos,frame=lines,tabsize=4,obeytabs=true,fontsize=\footnotesize
			]{c}{source_codes/newton.c}

\end{apendicesenv}
