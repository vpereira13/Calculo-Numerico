% ---
% Inicia os apêndices
% ---
\begin{apendicesenv}

% Imprime uma página indicando o início dos apêndices
\partapendices

\chapter{Códigos Fonte}
\label{ch:codigos_fonte}
	\section{Programas principais}
		\subsection*{\texttt{main\_simpson.c}}
		\label{app:main_simpson.c}
		\inputminted[autogobble,breaklines,linenos,frame=lines,tabsize=4,obeytabs=true,fontsize=\footnotesize
		]{c}{source_codes/main_simpson.c}

		\subsection*{\texttt{main\_newton.c}}
		\label{app:main_newton.c}
		\inputminted[autogobble,breaklines,linenos,frame=lines,tabsize=4,obeytabs=true,fontsize=\footnotesize
		]{c}{source_codes/main_newton.c}

	\section{Programas auxiliares}
		\subsection*{\texttt{Makefile}}
		\label{app:Makefile}
			\inputminted[autogobble,breaklines,linenos,frame=lines,tabsize=4,obeytabs=true,fontsize=\footnotesize
			]{Makefile}{source_codes/Makefile}

		\subsection*{\emph{Header} da biblioteca de Funções (\texttt{funcoes.h})}
		\label{app:funcoes.h}
			\inputminted[autogobble,breaklines,linenos,frame=lines,tabsize=4,obeytabs=true,fontsize=\footnotesize
			]{c}{source_codes/funcoes.h}

		\subsection*{Implementação da biblioteca de Funções (\texttt{funcoes.c})}
			\label{app:funcoes.c}
			\inputminted[autogobble,breaklines,linenos,frame=lines,tabsize=4,obeytabs=true,fontsize=\footnotesize
			]{c}{source_codes/funcoes.c}

		\subsection*{\emph{Header} da biblioteca de Simpson (\texttt{simpson.h})}
		\label{app:simpson.h}
			\inputminted[autogobble,breaklines,linenos,frame=lines,tabsize=4,obeytabs=true,fontsize=\footnotesize
			]{c}{source_codes/simpson.h}

		\subsection*{Implementação da biblioteca de Simpson (\texttt{simpson.c})}
			\label{app:simpson.c}
			\inputminted[autogobble,breaklines,linenos,frame=lines,tabsize=4,obeytabs=true,fontsize=\footnotesize
			]{c}{source_codes/simpson.c}

		\subsection*{\emph{Header} da biblioteca de Newton (\texttt{newton.h})}
		\label{app:newton.h}
			\inputminted[autogobble,breaklines,linenos,frame=lines,tabsize=4,obeytabs=true,fontsize=\footnotesize
			]{c}{source_codes/newton.h}

		\subsection*{Implementação da biblioteca de Newton (\texttt{newton.c})}
			\label{app:newton.c}
			\inputminted[autogobble,breaklines,linenos,frame=lines,tabsize=4,obeytabs=true,fontsize=\footnotesize
			]{c}{source_codes/newton.c}


\chapter{Arquivos de entrada}
\label{ch:inputs}
	\section{Método de Simpson 1/3 composto}
		\subsection*{Intervalo \([0, 1]\)}
			\label{app:input_simpson1}
			\inputminted[autogobble,breaklines,linenos,frame=lines,tabsize=4,obeytabs=true,fontsize=\footnotesize
			]{text}{source_codes/simpson1.in}

		\subsection*{Intervalo \([0, 2]\)}
			\label{app:input_simpson2}
			\inputminted[autogobble,breaklines,linenos,frame=lines,tabsize=4,obeytabs=true,fontsize=\footnotesize
			]{text}{source_codes/simpson2.in}

	\section{Método de Newton}
		\label{app:input_newton}
		\inputminted[autogobble,breaklines,linenos,frame=lines,tabsize=4,obeytabs=true,fontsize=\footnotesize
		]{text}{source_codes/newton1.in}

\end{apendicesenv}
