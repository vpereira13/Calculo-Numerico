\chapter{Instruções de uso}
Nesse capítulo mostraremos como compilar e executar os programas que produzimos.
Fizemos dois programas, um para calular a integração usando o método de Simpson
1/3 composto e outro para calcular a raiz de uma função utilizando o método de
Newton. Porém os dois utilizam programas auxiliares, chamamos de bibliotecas,
para modularizar nosso trabalho.

\section{Arquivos principais}
Os arquivos que produzimos são:
	\begin{itemize}
		\texttt{funcoes.c}
		\texttt{funcoes.h}
		\texttt{newton.c}
		\texttt{newton.h}
		\texttt{simpson.c}
		\texttt{simpson.h}
		\texttt{main_newton.c}
		\texttt{main_simpson.c}
		\texttt{Makefile}
	\end{itemize}

\section{Compilando}
Para compilar os dois programas utilizaremos a ferramenta \texttt{make}, que
utiliza como base o arquivo \texttt{Makefile}. Para compilar os dois programas
basta estar no diretório dos arquivos e rodar os seguintes comandos:

\begin{minted}[autogobble,breaklines,lineos,frame=lines,fontsize=\footnotesize]{bash}
	make main_newton	#Para compilar o programa que utiliza o método de Newton
	make main_simpson	#Para compilar o programa que utiliza o método de
Simpson 1/3
\end{minted}

\section{Rodando}
Para executar os programas há duas formas, automática, a qual é preciso um
arquivo de entrada, ou colocando os dados um a um. Como foi solicitado uns
testes específicos, deixamos disponíveis, na pasta \texttt{inputs}, os testes
para serem executados.

\subsection{Executando programa sem arquivo de entrada}
\begin{minted}[autogobble,breaklines,lineos,frame=lines,fontsize=\footnotesize]{bash}
	make run_newton		#Para executar o método de Newton
	make run_simpson	#Para executar o método de Simpson
\end{minted}

\subsection{Executando programa com arquivo de entrada}
\begin{minted}[autogobble,breaklines,lineos,frame=lines,fontsize=\footnotesize]{bash}
	make run_newton	< arquivo_de_input	#Para executar o método de Newton
	make run_simpson < arquivo_de_input	#Para executar o método de Simpson
\end{minted}

