% ---
% Inicia os apêndices
% ---
\begin{apendicesenv}

% Imprime uma página indicando o início dos apêndices
\partapendices

\chapter{Códigos Fonte}

	\section{Programa principal (\texttt{main.c})}
		\inputminted[autogobble,breaklines,linenos,frame=lines,tabsize=4,obeytabs=true,fontsize=\footnotesize
		]{c}{source_codes/main.c}
		\label{lst:main.c}

	\section{Biblioteca auxiliar}
		\subsection{\emph{Header} da biblioteca (\texttt{gauss\_seidel.h})}
			\inputminted[autogobble,breaklines,linenos,frame=lines,tabsize=4,obeytabs=true,fontsize=\footnotesize
			]{c}{source_codes/gauss_seidel.h}
			\label{lst:gauss_seidel.h}

		\subsection{Implementação da biblioteca (\texttt{gauss\_seidel.c})}
			\inputminted[autogobble,breaklines,linenos,frame=lines,tabsize=4,obeytabs=true,fontsize=\footnotesize
			]{c}{source_codes/gauss_seidel.c}
			\label{lst:gauss_seidel.c}

	\section{Programa auxiliar (\texttt{gera\_input.c})}
		\inputminted[autogobble,breaklines,linenos,frame=lines,tabsize=4,obeytabs=true,fontsize=\footnotesize
			]{c}{source_codes/gera_input.c}
			\label{lst:gera_input.c}

\end{apendicesenv}