\chapter{Método Iterativo de Gauss-Seidel}
O método de Gauss-Seidel é um método iterativo baseado no método de Jacobi desenvolvido para resolver sistemas de equações lineares. Seu nome foi originado nos matemáticos Carl Friedrich Gauss e Philipp Ludwig von Seidel.

Diferente do seu precursor, o método de Gauss-Seidel se aproveita de resultados de iterações passadas para acelerar a convergência para a resposta do sistema linear.

Podemos fazer um comparativo visual entre as duas funções de iteração, primeiramente a de Jacobi:

\[x_i^{(k+1)} = \dfrac{1}{a_{ii}}\left(b_i - \sum_{j=1\\j\neq i}^{n}{a_{ij}x_{j}^{(k)}}\right), \quad i = 1, 2, \ldots, n\]

Já o método de Gauss-Seidel possui a seguinte fórmula de iteração:

\[x_i^{(k+1)} = \dfrac{1}{a_{ii}}\left(b_i - \sum_{j=1}^{i-1}{a_{ij}x^{(k+1)}} - \sum_{j=i+1}^{n}{a_{ij}x_{j}^{(k)}}\right), \quad i = 1, 2, \ldots, n\]

Assim podemos ver que por se utilizar das iterações anteriores, onde  o método proposto para esse trabalho converge mais rápido que o de Jacobi.

Esses dois métodos são indicados para aplicações onde a matriz \(A\) é esparsa, ou seja, quando possui uma grande quantidade de elementos que valem zero. Para a utilização do método, temos que aplicar o critério de Sassenfeld e, claro, a matriz \(A\) tem que ser não-singular.