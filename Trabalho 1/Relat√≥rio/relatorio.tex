%%%%%%%%%%%%%%%%%%%%%%%%%%%%%%%%%%%%%%%%%%%%%%%%%%%%%%%%%%%%%%%%%%%%%
%																	%
%	Trabalho 1 -  Métodos Iterativos para  Sitemas Lineares			%
%																	%
%	Cálculo Numérico	SME-0104									%
%	Prof.: Murilo Francisco Tomé									%
%																	%
%	Eduardo Brunaldi dos Santos				8642515					%
%	Jorge Ashkar Ferreira Simondi			8517081					%
%	Victor Luiz da Silva Mariano Pereira	8602444					%
%																	%
%%%%%%%%%%%%%%%%%%%%%%%%%%%%%%%%%%%%%%%%%%%%%%%%%%%%%%%%%%%%%%%%%%%%%

\documentclass[
	% -- opções da classe memoir --
	article,			% indica que é um artigo acadêmico
	11pt,				% tamanho da fonte
	twoside,			% para impressão frente e verso
	openany,
	a4paper,			% tamanho do papel.
	sumario=tradicional
	]{abntex2}

% ---
% PACOTES
% ---

% ---
% Pacotes fundamentais
% ---
\usepackage{lmodern}			% Usa a fonte Latin Modern
\usepackage{color}				% Controle das cores
\usepackage{graphicx}			% Inclusão de gráficos
	\graphicspath{Figuras/}
\usepackage{polyglossia}
	\setdefaultlanguage{portuges}
\usepackage{microtype} 			% para melhorias de justificação
% ---

% ---
% Pacotes para fins matemáticos
% ---
\usepackage{amsmath}
\usepackage{amssymb}
\usepackage{amsfonts}
\usepackage{amsthm}
% ---

% ---
% Pacotes de citações
% ---
\usepackage[brazilian,hyperpageref]{backref}	% Paginas com as citações na bibl
\usepackage[alf]{abntex2cite}					% Citações padrão ABNT
\usepackage{caption}							% Pacote para resolver o problema do
												% "listings", "minted" e códigos muito grandes
\usepackage{listings}							% Pacote para listagens de códigos
\usepackage{minted}								% Inserção de códigos fonte
% ---

% ---
% Pacotes adicionais
% ---
\usepackage{showframe}
\usepackage{multicol}
% ---

% ---
% Informações de dados para CAPA e FOLHA DE ROSTO
% ---
\titulo{Trabalho 1\\
		Métodos Iterativos para Sistemas Lineares}
\autor{
		Eduardo Brunaldi dos Santos\\ \small{8642515} \and
		Jorge Ashkar Ferreira Simondi\\ \small{ 8517081} \and
		Victor Luiz da Silva Mariano Pereira\\ \small{8602444}
	}
\local{Brasil}
\data{2018}
% ---

% ---
% Configurações de aparência do PDF final

% alterando o aspecto da cor azul
\definecolor{blue}{RGB}{41,5,195}

% informações do PDF
\makeatletter
\hypersetup{
     	%pagebackref=true,
		pdftitle={\@title},
		pdfauthor={\@author},
    	pdfsubject={Método Iterativos Sistemas Lineares},
	    pdfcreator={LaTeX with abnTeX2 and LuaLaTeX},
		pdfkeywords={Cáculo Numérico}{ Métodos Iterativos}{ Gauss-Seidel},
		colorlinks=true,       		% false: boxed links; true: colored links
    	linkcolor=blue,          	% color of internal links
    	citecolor=blue,        		% color of links to bibliography
    	filecolor=magenta,      	% color of file links
		urlcolor=blue,
		bookmarksdepth=4
}
\makeatother
% ---

% ---
% compila o indice
% ---
\makeindex
% ---

% ---
% Altera as margens padrões
% ---
\setlrmarginsandblock{3cm}{3cm}{*}
\setulmarginsandblock{3cm}{3cm}{*}
\checkandfixthelayout
% ---

% ---
% Espaçamentos entre linhas e parágrafos
% ---

% O tamanho do parágrafo é dado por:
\setlength{\parindent}{1.3cm}

% Controle do espaçamento entre um parágrafo e outro:
\setlength{\parskip}{0.2cm}  % tente também \onelineskip

% Espaçamento simples
\SingleSpacing

% ----
% Início do documento
% ----
\begin{document}

% Retira espaço extra obsoleto entre as frases.
\frenchspacing

% ----------------------------------------------------------
% ELEMENTOS PRÉ-TEXTUAIS
% ----------------------------------------------------------

%---
% página de titulo
\maketitle

% ---

% ----------------------------------------------------------
% ELEMENTOS TEXTUAIS
% ----------------------------------------------------------
\textual

% ----------------------------------------------------------
% Introdução
% ----------------------------------------------------------
\chapter*[Introdução]{Introdução}
\addcontentsline{toc}{chapter}{Introdução}
Em cálculo, algumas integrais definidas não são tão simples de calcular, para
isso foram desenvolvidos métodos numéricos para calcular essas integrais
difíceis. As integrações numéricas são formas para aproximar uma intragral que
tenha a caracteristica ou até mesmo quando não se tem a função propriamente
dita, mas tem alguns valores da função em alguns pontos.

No caso desse segundo trabalho de Cálculo Numérico, tivemos que implementar um
programa em Matlab/Octave ou em C que calculasse a integral de uma dada função
utilizando o método de Simpson 1/3 composto.

Também precisamos fazer alguns testes solicitados para porvar a existência de
raiz num certo intervalo, a partir dos resultados encontrados na implementação e
teste do método.


% ----------------------------------------------------------
% Seção de explicações
% ----------------------------------------------------------
\chapter{Integração numérica Simpson 1/3 composta}


% ----------------------------------------------------------
% Seção de códigos fonte
% ----------------------------------------------------------
% ---
% Inicia os apêndices
% ---
\begin{apendicesenv}

% Imprime uma página indicando o início dos apêndices
\partapendices

\chapter{Códigos Fonte}
\label{ch:codigos_fonte}
	\section{Programa principal (\texttt{main.c})}
		\label{sec:main.c}
		\inputminted[autogobble,breaklines,linenos,frame=lines,tabsize=4,obeytabs=true,fontsize=\footnotesize
		]{c}{source_codes/main.c}

	\section{Biblioteca auxiliar}
		\subsection{\emph{Header} da biblioteca (\texttt{simpson\_e\_newton.h})}
		\label{sec:simpson_composta.h}
			\inputminted[autogobble,breaklines,linenos,frame=lines,tabsize=4,obeytabs=true,fontsize=\footnotesize
			]{c}{source_codes/simpson_e_newton.h}

		\subsection{Implementação da biblioteca (\texttt{simpson\_e\_newton.c})}
			\label{sec:simpson_composta.c}
			\inputminted[autogobble,breaklines,linenos,frame=lines,tabsize=4,obeytabs=true,fontsize=\footnotesize
			]{c}{source_codes/simpson_e_newton.c}

\end{apendicesenv}


% ----------------------------------------------------------
% Seção de resultados dos testes
% ----------------------------------------------------------
\chapter{Resultados}
Apesar da precisão \(\epsilon\) ser especificada em alguns casos, em outros
não, o programa que nós fizemos sempre vai imprimir o resultado encontrado
com 16 casas decimais. Outro ponto a se observar, é que na apresentação dos
resultados mostraremos os componentes do vetor \(x\) obtidos, fizemos isso
por questão de estética e ter que colocar um vetor grande com 50 a 100
elementos.

\section{Teste 1}
Nesse primeiro teste, foi solicitado que executássemos nosso programa com os seguintes dados:

	\begin{itemize}
		\item \(n = 50\)
		\item Regra de formação de \(A\) é denotada por:
		\[
			\begin{cases}
				a_{i,i} = 4,	& i = 1, 2, \ldots, n;\\
				a_{i,i+1} = -1,	& i = 1, 2, \ldots, n-1;\\
				a_{i+1,i} = -1,	& i = 1, 2, \ldots, n-1;\\
				a_{i,i+3} = -1,	& i = 1, 2, \ldots, n-3;\\
				a_{i+3,i} = -1,	& i = 1, 2, \ldots, n-3;\\
				a_{i,j} = 0,	& \mbox{no restante.}
			\end{cases}
		\]
		\item \(b_{i} = \sum_{j=1}^{n} a_{ij}, \quad i = 1, 2, \ldots, n\)
		\item A tolerância de erro e a quantidade de iterações máxima não foram especificadas,
			então deixamos como padrão \(\epsilon = 10^{-5}\) e \(itmax = 10^{7}\).
	\end{itemize}
Com 429 iterações, conseguimos o seguinte vetor \(x\) como resultado:
\begin{multicols}{3}
	\begin{align*}
		x_{1}  = 0.9999337056119755\\
		x_{2}  = 0.9999022580617377\\
		x_{3}  = 0.9998767564132660\\
		x_{4}  = 0.9998373110649970\\
		x_{5}  = 0.9998043818757397\\
		x_{6}  = 0.9997745311874137\\
	\end{align*}
	\vfill
	\columnbreak
	\begin{align*}
		x_{7}  = 0.9997426561581359\\
		x_{8}  = 0.9997127717351563\\
		x_{9}  = 0.9996847662462976\\
		x_{10} = 0.9996574884419451\\
		x_{11} = 0.9996317401867282\\
		x_{12} = 0.9996076670815898\\
	\end{align*}
	\vfill
	\columnbreak
	\begin{align*}
		x_{13} = 0.9995850658671707\\
		x_{14} = 0.9995641285500906\\
		x_{15} = 0.9995449525304450\\
		x_{16} = 0.9995275349651083\\
		x_{17} = 0.9995119448425527\\
		x_{18} = 0.9994982360680523\\
	\end{align*}
\end{multicols}
\begin{multicols}{3}
	\begin{align*}
		x_{19} = 0.9994864324485824\\
		x_{20} = 0.9994765653157160\\
		x_{21} = 0.9994686592137438\\
		x_{22} = 0.9994627261862783\\
		x_{23} = 0.9994587736375708\\
		x_{24} = 0.9994568022326934\\
		x_{25} = 0.9994568043937440\\
		x_{26} = 0.9994587659437409\\
		x_{27} = 0.9994626656965930\\
		x_{28} = 0.9994684755790514\\
		x_{29} = 0.9994761612548965\\
	\end{align*}
	\columnbreak
	\vfill
	\begin{align*}
		x_{30} = 0.9994856808399818\\
		x_{31} = 0.9994969866366955\\
		x_{32} = 0.9995100271367293\\
		x_{33} = 0.9995247400284760\\
		x_{34} = 0.9995410581501044\\
		x_{35} = 0.9995589220334093\\
		x_{36} = 0.9995782457760340\\
		x_{37} = 0.9995989330477007\\
		x_{38} = 0.9996209526818133\\
		x_{39} = 0.9996441831497855\\
		x_{40} = 0.9996684275186901\\
	\end{align*}
	\columnbreak
	\vfill
	\begin{align*}
		x_{41} = 0.9996938378919464\\
		x_{42} = 0.9997202487630036\\
		x_{43} = 0.9997469483563225\\
		x_{44} = 0.9997749237180661\\
		x_{45} = 0.9998042064202979\\
		x_{46} = 0.9998313692296440\\
		x_{47} = 0.9998608077401657\\
		x_{48} = 0.9998952663788790\\
		x_{49} = 0.9999175540360030\\
		x_{50} = 0.9999445904440422\\
		\columnbreak
	\end{align*}
\end{multicols}
\section{Teste 2}
No segundo teste, foi solicitado que executássemos nosso programa com os seguintes dados:

	\begin{itemize}
		\item \(n = 100\)
		\item Regra de formação de \(A\) é denotada por:
		\[
			\begin{cases}
				a_{i,i} = 4,	& i = 1, 2, \ldots, n;\\
				a_{i,i+1} = -1,	& i = 1, 2, \ldots, n-1;\\
				a_{i+1,i} = -1,	& i = 1, 2, \ldots, n-1;\\
				a_{i,i+3} = -1,	& i = 1, 2, \ldots, n-3;\\
				a_{i+3,i} = -1,	& i = 1, 2, \ldots, n-3;\\
				a_{i,j} = 0,	& \mbox{no restante.}
			\end{cases}
		\]
		\item \(b_{i} = \sum_{j=1}^{n} a_{ij}, \quad i = 1, 2, \ldots, n\)
		\item A tolerância de erro e a quantidade de iterações máxima não foram especificadas,
			então deixamos como padrão \(\epsilon = 10^{-5}\) e \(itmax = 10^{7}\).
	\end{itemize}
Com 1357 iterações, conseguimos o seguinte vetor \(x\) como resultado:
\begin{multicols}{3}
	\begin{align*}
		x_{1}  = 0.9998738542396985\\
	\end{align*}
	\vfill
	\columnbreak
	\begin{align*}
		x_{2}  = 0.9998129435623880\\
	\end{align*}
	\vfill
	\columnbreak
	\begin{align*}
		x_{3}  = 0.9997626042371170\\
	\end{align*}
	\vfill
	\columnbreak
	\begin{align*}
		x_{4}  = 0.9996848479263462\\
		x_{5}  = 0.9996182432345047\\
		x_{6}  = 0.9995562138527778\\
		x_{7}  = 0.9994889115229971\\
		x_{8}  = 0.9994237903157294\\
		x_{9}  = 0.9993605364284126\\
		x_{10} = 0.9992968430669336\\
		x_{11} = 0.9992341148124684\\
		x_{12} = 0.9991725355247364\\
		x_{13} = 0.9991116311073441\\
		x_{14} = 0.9990516865321048\\
		x_{15} = 0.9989928222278777\\
		x_{16} = 0.9989349787055476\\
		x_{17} = 0.9988782439532933\\
		x_{18} = 0.9988226902835039\\
		x_{19} = 0.9987683448745465\\
		x_{20} = 0.9987152612888901\\
		x_{21} = 0.9986634928002946\\
		x_{22} = 0.9986130812852338\\
		x_{23} = 0.9985640720869986\\
		x_{24} = 0.9985165101388453\\
		x_{25} = 0.9984704367415823\\
		x_{26} = 0.9984258924917748\\
		x_{27} = 0.9983829168228631\\
		x_{28} = 0.9983415471628708\\
		x_{29} = 0.9983018194640830\\
		x_{30} = 0.9982637681626284\\
		x_{31} = 0.9982274259594643\\
		x_{32} = 0.9981928238934358\\
		x_{33} = 0.9981599913289189\\
		x_{34} = 0.9981289558893802\\
		x_{35} = 0.9980997434513928\\
		x_{36} = 0.9980723781293944\\
	\end{align*}
	\vfill
	\columnbreak
	\begin{align*}
		x_{37} = 0.9980468822474435\\
		x_{38} = 0.9980232763240508\\
		x_{39} = 0.9980015790579967\\
		x_{40} = 0.9979818073123535\\
		x_{41} = 0.9979639761023460\\
		x_{42} = 0.9979480985851344\\
		x_{43} = 0.9979341860505684\\
		x_{44} = 0.9979222479140781\\
		x_{45} = 0.9979122917114375\\
		x_{46} = 0.9979043230951400\\
		x_{47} = 0.9978983458326130\\
		x_{48} = 0.9978943618062428\\
		x_{49} = 0.9978923710151436\\
		x_{50} = 0.9978923715787057\\
		x_{51} = 0.9978943597419185\\
		x_{52} = 0.9978983298824433\\
		x_{53} = 0.9979042745194355\\
		x_{54} = 0.9979121843241067\\
		x_{55} = 0.9979220481320004\\
		x_{56} = 0.9979338529569851\\
		x_{57} = 0.9979475840069450\\
		x_{58} = 0.9979632247011112\\
		x_{59} = 0.9979807566890773\\
		x_{60} = 0.9980001598714995\\
		x_{61} = 0.9980214124222428\\
		x_{62} = 0.9980444908122079\\
		x_{63} = 0.9980693698350556\\
		x_{64} = 0.9980960226336919\\
		x_{65} = 0.9981244207284375\\
		x_{66} = 0.9981545340486365\\
		x_{67} = 0.9981863309622862\\
		x_{68} = 0.9982197783064963\\
		x_{69} = 0.9982548414300648\\
	\end{align*}
	\vfill
	\columnbreak
	\begin{align*}
		x_{70} = 0.9982914842231527\\
		x_{71} = 0.9983296691384520\\
		x_{72} = 0.9983693572692299\\
		x_{73} = 0.9984105083744410\\
		x_{74} = 0.9984530808251019\\
		x_{75} = 0.9984970318264450\\
		x_{76} = 0.9985423174643067\\
		x_{77} = 0.9985888922234198\\
		x_{78} = 0.9986367098011544\\
		x_{79} = 0.9986857235278032\\
		x_{80} = 0.9987358836422807\\
		x_{81} = 0.9987871403698153\\
		x_{82} = 0.9988394474634000\\
		x_{83} = 0.9988927483413111\\
		x_{84} = 0.9989469867000615\\
		x_{85} = 0.9990021309945048\\
		x_{86} = 0.9990581078225971\\
		x_{87} = 0.9991148304313648\\
		x_{88} = 0.9991723487039626\\
		x_{89} = 0.9992305448790375\\
		x_{90} = 0.9992891527897000\\
		x_{91} = 0.9993486065323530\\
		x_{92} = 0.9994087268533902\\
		x_{93} = 0.9994682235509935\\
		x_{94} = 0.9995292193590008\\
		x_{95} = 0.9995919759926195\\
		x_{96} = 0.9996496775214918\\
		x_{97} = 0.9997113652847045\\
		x_{98} = 0.9997829628126369\\
		x_{99} = 0.9998293169924293\\
		x_{100} = 0.9998851705692834\\
	\end{align*}
	\vfill
\end{multicols}

\section{Teste 3}
Já no terceiro teste solicitado, diferente dos anteriores, foram especificados
os valores de \(\epsilon\) e de \(itmax\), ficando da seguinte maneira:

	\begin{itemize}
		\item \(n = 100\)
		\item Regra de formação de \(A\) é denotada por:
		\[
			\begin{cases}
				a_{i,i} = 4,	& i = 1, 2, \ldots, n;\\
				a_{i,i+1} = -1,	& i = 1, 2, \ldots, n-1;\\
				a_{i+1,i} = -1,	& i = 1, 2, \ldots, n-1;\\
				a_{i,i+3} = -1,	& i = 1, 2, \ldots, n-3;\\
				a_{i+3,i} = -1,	& i = 1, 2, \ldots, n-3;\\
				a_{i,j} = 0,	& \mbox{no restante.}
			\end{cases}
		\]
		\item \(b_{i} = 1.0/i, \quad i = 1, 2, \ldots, n\)
		\item \(\epsilon = 10^{-10}\)
		\item \(itmax = 10^{7}\)
	\end{itemize}
Com 4035 iterações, conseguimos o seguinte vetor \(x\) como resultado:
\begin{multicols}{3}
	\begin{align*}
		x_{1}  = 0.9187448165907161\\
		x_{2}  = 1.1045965699783248\\
		x_{3}  = 1.2325874938164970\\
		x_{4}  = 1.5703826964083346\\
		x_{5}  = 1.7670539695354222\\
		x_{6}  = 1.9220373755819542\\
		x_{7}  = 2.1131445057330069\\
		x_{8}  = 2.2711992362214035\\
		x_{9}  = 2.4086968666307383\\
		x_{10} = 2.5461015719238098\\
		x_{11} = 2.6709016030530332\\
		x_{12} = 2.7843381717571969\\
		x_{13} = 2.8916633123568407\\
		x_{14} = 2.9910583414847695\\
	\end{align*}
	\vfill
	\columnbreak
	\begin{align*}
		x_{15} = 3.0827575717446547\\
		x_{16} = 3.1682320874339032\\
		x_{17} = 3.2474823074568488\\
		x_{18} = 3.3207350197423259\\
		x_{19} = 3.3885251582887751\\
		x_{20} = 3.4510802518712860\\
		x_{21} = 3.5086194860451511\\
		x_{22} = 3.5614216952868021\\
		x_{23} = 3.6096940558257294\\
		x_{24} = 3.6536219297971451\\
		x_{25} = 3.6933935356735150\\
		x_{26} = 3.7291740856229907\\
		x_{27} = 3.7611139751266089\\
		x_{28} = 3.7893564321402836\\
	\end{align*}
	\vfill
	\columnbreak
	\begin{align*}
		x_{29} = 3.8140332375616510\\
		x_{30} = 3.8352664160698222\\
		x_{31} = 3.8531706946495210\\
		x_{32} = 3.8678532579607074\\
		x_{33} = 3.8794144237793418\\
		x_{34} = 3.8879486080854190\\
		x_{35} = 3.8935446760295326\\
		x_{36} = 3.8962863828783725\\
		x_{37} = 3.8962528733599743\\
		x_{38} = 3.8935190268071943\\
		x_{39} = 3.8881557807540906\\
		x_{40} = 3.8802304488311668\\
		x_{41} = 3.8698069878026507\\
		x_{42} = 3.8569462390514479\\
	\end{align*}
	\vfill
	\columnbreak
	\begin{align*}
		x_{43} = 3.8417061536023001\\
		x_{44} = 3.8241419928140214\\
		x_{45} = 3.8043065104340237\\
		x_{46} = 3.7822501199568167\\
		x_{47} = 3.7580210468883337\\
		x_{48} = 3.7316654678906484\\
		x_{49} = 3.7032276386672660\\
		x_{50} = 3.6727500113467956\\
		x_{51} = 3.6402733424392654\\
		x_{52} = 3.6058367924089371\\
		x_{53} = 3.5694780175913818\\
		x_{54} = 3.5312332551719267\\
		x_{55} = 3.4911374019048185\\
		x_{56} = 3.4492240871063841\\
		x_{57} = 3.4055257404294229\\
		x_{58} = 3.3600736549446078\\
		x_{59} = 3.3128980458356888\\
		x_{60} = 3.2640281050369084\\
		x_{61} = 3.2134920524875769\\
		x_{62} = 3.1613171838998899\\
	\end{align*}
	\vfill
	\columnbreak
	\begin{align*}
		x_{63} = 3.1075299149135004\\
		x_{64} = 3.0521558236288166\\
		x_{65} = 2.9952196902850702\\
		x_{66} = 2.9367455313930225\\
		x_{67} = 2.8767566370039185\\
		x_{68} = 2.8152756070057247\\
		x_{69} = 2.7523243683867061\\
		x_{70} = 2.6879242130190810\\
		x_{71} = 2.6220958501560915\\
		x_{72} = 2.5548593686640197\\
		x_{73} = 2.4862342823983469\\
		x_{74} = 2.4162397050438427\\
		x_{75} = 2.3448940849725689\\
		x_{76} = 2.2722152128718184\\
		x_{77} = 2.1982210892729214\\
		x_{78} = 2.1229287201109216\\
		x_{79} = 2.0463535002357254\\
		x_{80} = 1.9685137062025785\\
		x_{81} = 1.8894256932616957\\
		x_{82} = 1.8090981340243647\\
	\end{align*}
	\vfill
	\columnbreak
	\begin{align*}
		x_{83} = 1.7275545421499002\\
		x_{84} = 1.6448165338014529\\
		x_{85} = 1.5608636785986605\\
		x_{86} = 1.4757416018946408\\
		x_{87} = 1.3895174593800130\\
		x_{88} = 1.3020337388756733\\
		x_{89} = 1.2134028205516427\\
		x_{90} = 1.1239837101476174\\
		x_{91} = 1.0329873606760617\\
		x_{92} = 0.9406162762941549\\
		x_{93} = 0.8489160889277005\\
		x_{94} = 0.7543267064481112\\
		x_{95} = 0.6562892698483521\\
		x_{96} = 0.5659849746877800\\
		x_{97} = 0.4684758085009985\\
		x_{98} = 0.3537028062007704\\
		x_{99} = 0.2798420648291652\\
		x_{100} = 0.1895794683325409
	\end{align*}
\end{multicols}


% ---
% Conclusão
% ---
\chapter*{Conclusão}
\addcontentsline{toc}{chapter}{Conclusão}
Como foi pedido, utilizamos a linguagem C como base para fazer o trabalho. Implementamos também algumas bibliotecas para nos auxiliar e modularizar o código, contendo métodos específicos de cada regra, Simpson 1/3 e de Newton, e uma de base para as funções que precisávamos calcular.

Na prática, após estudar a matéria, não tivemos tantas dificuldades sobre como resolver o problema proposto, o grupo conseguiu implementar os códigos de Simpson 1/3 e de Newton e executá-los com as entradas necessárias para obter os resultados com uma boa precisão.

Para a realização dos testes pedidos, utilizamos arquivos de entrada que podem ser vistos no apêndice \ref{ch:inputs}, nos quais os utilizados para a regra de Simpson 1/3 composta fizemos com 10000 subdivisões, para obter uma boa precisão, já o utilizado para o método de Newton, está como foi pedido, precisão de \(10^{-10}\) e com chute inicial \(x_0 = 0.5\).

Com relação aos resultados obtidos no capítulo~\ref{sec:resultadoSimpson} podemos verificar a eficiência da regra de Simpson 1/3 composta e, também, do nosso trabalho.

% ----------------------------------------------------------
% ELEMENTOS PÓS-TEXTUAIS
% ----------------------------------------------------------
\postextual

\end{document}
