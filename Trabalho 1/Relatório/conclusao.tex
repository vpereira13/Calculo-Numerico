\chapter*{Conclusão}
\addcontentsline{toc}{chapter}{Conclusão}
Como pode se ver no (anexo A), nós utilizamos a linguagem C como base para fazer o trabalho, implementamos também uma pequena biblioteca para nos auxiliar na modularização de algumas funções.

Na prática, após estudar a matéria, não tivemos tantas dúvidas sobre como resolver o problema proposto. Por fim, o grupo conseguiu implementar o código do método de Gauss-Seidel e executá-lo com as entradas necessárias para se obter a aproximação desejada.

Para os testes, usamos um outro código desenvolvido pelo grupo para automatizar a criação das entradas, que também pode ser visto no (anexo A).

Com relação aos resultados, foi possível verificar que o método utilizado realiza uma aproximação em menor número de iterações comparado ao método de Jacobi, que os resultados com o mesmo grau de aproximação gastaria praticamente o dobro de iterações.